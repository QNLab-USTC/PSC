\documentstyle{article}
\begin{document}
We use a discrete-event-based quantum network simulation platform SimQN. The LP solver used in our simulator is the GEKKO Optimization suite package. Regarding network topology construction in the experiments, we use a random topology generation method based on a minimum spanning tree to generate an arbitrary quantum network topology with $N$ quantum nodes, which are connected randomly with $1.5N$ quantum links. By default, the network has 100 quantum nodes and 20 quantum requests. The initial fidelity of the entangled pairs generated using quantum links, $F_0$, is between [0.90,0.95], and the capacity of the quantum links is 100 (we can generate up to 100 entanglement resources between adjacent nodes). We also set the threshold $F^{*} = 0.8$ of the required fidelity of each request to 0.8. In addition, we use two schemes as the comparison scheme. The first is the Propagatory Update (PU) algorithm which has performed better for resource allocation, and it uses the threshold-based ($F_{th}$) purification strategy (in our experiments, we set $F_{th}$ to 0.925). We run each set of parameters 200 times to obtain statistical expectations for each performance metric and reduce randomness.
\end{document}